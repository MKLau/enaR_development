


The network approach has been quite useful in ecosystem ecology, in
which many core questions center on how species are connected across
the intricate web of interactions.  Ecosystem Network Analysis (ENA)
has been successfully used to address a range of key questions.  For
example, \citet{bondavalli1999} found evidence of an indirect
mutualism between the American alligator and several of its prey
items, including several invertebrates, frogs, and small mammals in
the food web of Big Cypress National Preserve (Florida, USA).
Further, appliation of ENA has lead to new insights into the classic
question of ``What limits food-chain length?'' \citep{ulanowicz2014}.
From a different perspective, \citep{hines12} used ENA to show the
relative importance of coupling of biogeochemical processes (e.g.,
nitrification + anammox) in the Cape Fear River estuary sediementary
nitrogen cycle.  Further, scientists have used ENA to investigate
differences in urban sustainability \citep{bodini02, zhang10_ecomod,
  chen12, bodini2012cities}.  Collectively, this work consistently
shows the power of a transactional network to generate unexpected
ecological relationships that then influence the system function and
evolution \citep{ulanowicz97, patten91, jorgensen07_newecology}.

