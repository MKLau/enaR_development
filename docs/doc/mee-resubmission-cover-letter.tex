\documentclass[letter]{letter}
\usepackage[]{uncwletter}
\newcommand{\R}{R}
\usepackage{url}

\name{Stuart R. Borrett\\Associate Professor}
\department{Dept. Biology and Marine Biology}
%\location{Dobo Hall}
%\mailcode{}
%\telephone{910-962-2411}
%\email{borretts\@uncw.edu}

\begin{document}

\begin{letter}{Professor Rob Freckleton\\
Executive Editor\\
Methods in Ecology and Evolution\\
British Ecological Society
}

\opening{Professor Freckleton, }

With this letter, my coauthor Matt Lau and I are submitting a revised
version of our article entitled \emph{\texttt{enaR}: An \R\ Package
for Ecosystem Network Analysis} for reconsideration and publication in
\emph{Methods in Ecology and Evolution} as an Application paper.

With our manuscirpt revisions, we have addressed the main critiques by
the associate editor and two initial reviewers.  We appreciate their inital
evaluation of the work and help imporoving the final product.  In the
remander of this letter, we systematically identify how we addressed
the initial comments.

There were multiple comments to focus and clarify the introduction. In
response, we compretely rewrote the introduction to focus on the broad
MEE audience.  Specifically, we added text to make the ecological
motivation of the analyses more clear and reduced or eleminated the
text about the historical development of Ecosystem Network Analysis
and extant softare tools. We generally focused the text and reduced
the overall length be removing unecessary details.
In addition, throughout the manuscript we repalced the term
``Ecological Network Analysis'' with ``Ecosystem Network Analysis'' to
make the scope of the analyses more clear.  While ``Ecological
Network Analysis'' is the historical key term for these analyses, it
was clear in some of the comments that this generated confusion due to
the growth of network ecology.
tool
It was also suggested that we improve the clarity of how the package
is to be used and extended by the community. With this goal in mind,
we have added more ecological interpretation to the example analysis
text and annotation in the example, i.e., Table 4, which we have
updated so that it is reproducible. The ecological
interpretation of the example Flow Analysis is now presented in a
paragraph added to the Example Application sub-section. The previously
difficult to access, supplementary data file, oyster.dat, is no longer
required for the example. We also added text clarifying the data
requirements and inputs, model library contents, batch processing,
visualizations, and links to other packages to those respective
sub-sections or figure legends. Missing references in the description
of the example results have also been added.  Further, as online
supplimentary material we have included an annotated version of the
\R\ script that generates Figure 1, which we expect will help
potential users learn to use the package.

In response to several specific comments, we have expanded the
description of the package as a whole and the specific functions with
Tables 1, 2 and 3.  As the package includes 34 different functions
including nine main analysis functions that each perform multiple
subanalyses, it is not possible to extensively describe or illustrate
all of them in this short Application article.  Thus, we have worked
to provide enough detail to indentify the main types of analyses that
are available, supply ecological motivation for why they might be
useful, and to give a reference to additional literature that
interested readers can use to learn more about each analysis.

One concern was that our claim that the software was more available
and extinsible was not sufficently supported.  We have addressed this
in a couple of ways.  First, we have clarified that becuase this is an
\R\ package, all \R\ functions are available for review and local
editing by executing the \texttt{edit($function\_name$)} command.
Second, we have established a GitHub repository for enaR to manage the
collaborative development of this coding project.  This should enbable
a broad community to contribute to the package and view code directly
with all annotation visible in the code for each function.

We also revised and clarified some language in the paper as requested.  The text
stating that the package is "built on" existing packages has been
improved to explain that the package uses a data type that is defined
by and used in previously developed packages used for network analyses. We have
also added text describing the input data types and methods, partly
through Tables 1 through 3. Confusing text referring to the package as
professional grade has now been removed. Version numbering issues have
been removed by deleting specific reference to the version of the
package.  Further, if accepted this will be the first paper to introduce the
\R\ package to the ecological community.

Finally, throughout the manuscript we have attempted to clarify the
novilty of this software.  For the most part, the analyses and
algorithms in the package are not new to ecology; some of them are
more than 20 years old.  However, we see four key novel aspects to
this software.  First, the package collects the core algorithms for
Ecosystem Network Analysis from both the Patten and Ulanowicz schools
of development, which is uncommon amongst existing software.  Further,
it provides a software foundation that can continue to be expanded to
include new developments.  Second, due to the open source nature of
the chosen environments for execution (\R ) and developmnet (GitHub),
the algorithms are available for inspection and are adaptable and
extensible as the user needs.  Third, we have included a library of
100 empirically-based ecosystem models in the package to enable
cross-comparisons and new meta-analyses.  While these models have been
previouly published, they have not been collected and made available
in this way.  Fourth, because of the nature of the programming
environment and some of the software design decisions,
ecologists using this package have easy access to a range of
additional network analysis and statistical tools in \R .  Given these
novel features, we expect this software to contribute to the large and
rapidly growing area of network ecology and the domain of network
science in general.

Thank you in advance for your consideration of this revised article.

Respectfully,

\includegraphics[scale=0.7]{srb-sig}

Stuart R. Borrett
\end{letter}
\end{document}




%%% Outline with reviewer comments %%%

%%%Associate editor comments
% - Make more relevant to non-network people
% -- Done so by adding more ecology and background
% - Schools in intro is confusing
% -- Removed schools from intro and introduce later
% - Not enough clarity of use of the package
% -- Added more info on how the package is to be used
% - Try to be more pedagogic
% -- Added more ecology and scientific implications to intro
% -- Clarified text describing the use of the package in
% -- Added tables 1 to 3
% - Add a reproducible example
% -- Improved the clarity of the example Table 4
% -- Added model construction to the example
% -- removed the use of oyster.dat
% - Fail to describe how enaR could help me analyze data
% -- After reading the mvAbund and diversityR papers...
% -- Added ecosystem network analysis paragraph to intro
% -- Improved the clarity of the data requirements section
% -- Added more text on flow analysis implications/interpretation
% - Built on existing packages?
% -- Changed to text to explain that the data type used coheres to that
% used in other packages
% - Describe functions
% -- Add tables 1-3
% -- Added number of functions
% -- Added text on data types and formats
% -- Added text on data import methods
% - Not clear how this information should be considered in a simple
% graph (import/export)
% -- Added Fig 1 and associated text
% - Example not clear
% -- Added clearer walk through via annotation in script of Table 4
% -- Added more ecological interpretation of flow analysis in main body
% - Visualization, novelty compared to exisiting pacakge?
% -- ???
% - Does the library contain models or observed data?
% -- Added text in section ??? to clarify that the models are built on
% empirical data
% - Batch analysis not clear
% -- Clarified the use of the lapply function approach for batch
% analysis
% - How to interact with other packages?
% -- ???
% - Git repo?
% -- Added git info
% - Professional grade?
% -- Removed professional grade text

%%%Reviewer #1
% - Reduce intro length and clarify
% -- Added more ecology
% -- Removed schools
% -- Focused
% - Availability
% -- Added info for git repo
% - Pro-grade? Testing?
% -- Removed text
% -- Checked out the testthat package found it useful

%%%Reviewer #2
%- extensibility? guidence through the code
%-- github info
%-- added contributor info to github
%-- internal annotation visible on github
%- Version
%-- Removed specific reference to version
%- Specifiy recent developments
%-- Removed text for word economy
%- References?
%-- Added references to end of paragraph
%- Some indication of reproduction of previous results
%-- ???
%- Supplementary files
%-- oyster.rda?
