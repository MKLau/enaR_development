%\VignetteIndexEntry{ecological network analysis}
\documentclass[11pt]{article}
\usepackage[margin=1in]{geometry}
\bibliographystyle{ecology}
\usepackage[super, sort]{natbib}
  \bibpunct{(}{)}{;}{a}{,}{,} % required for natbib  
\usepackage{amssymb}
\usepackage{amsmath}
\usepackage{hyperref}  
\usepackage{ucs} %needed for R output: signif stars etc, quotes
\usepackage[utf8x]{inputenc}
\usepackage[T1]{fontenc}
\usepackage{sidecap}
\renewcommand{\floatpagefraction}{0.8}
\title{Vignette: enaR}
\author{S.R. Borrett and M.K. Lau}
%\date{}                                           % Activate to display a given date or no date

\def\R{\textsf{R}}
\def\tableline{\vskip .1in \hrule height 0.6pt \vskip 0.1in}

\usepackage{Sweave}
\begin{document}
\maketitle

\setcounter{tocdepth}{3}  %%activate to number sections
\tableofcontents

\section{Introduction}
This package is a collection of functions to implement Ecological
Network Analysis (ENA), which is a family of algorithms for
investigating the structure and function of ecosystems modeled as
networks of thermodynamically conserved energy--matter exchanges.  The
package brings together multiple ENA algorithms from several
approaches into one common software framework that is readily available and
extensible.  The package builds on the \textit{network} data structure
for \R\ developed by \citet{butts08_network}. In addition to being
able to perform several types of ENA with a single package, users can
also make use of network analysis tools built into the
\textit{network} package, the \textit{sna} (social network analysis)
package \citep{butts08_social}, and other components of what is now
called \textit{statnet} \citep{handcock2008statnet}.

This vignette illustrates how to use the \textit{enaR} package to
perform ENA.  It is not meant to be a detailed guide to ENA, but we
provide some references to the primary literature for those wishing to
learn more about the techniques.

\section{Background}
% Intro
Before describing how to use this package, we provide a brief
background of ENA.  Users may find this helpful as several software
design decisions were predicated on the history and current state of
the field.

% What is ENA
The ENA methodology is an application and extension of economic
Input--Output Analysis \citep{leontief1936,leontief66} that was first
introduced into ecology by \citet{hannon73}.  Two major schools have
developed in ENA.  The first is based on Dr.\ Robert E.\ Ulanowicz's
work with a strong focus on trophic dynamics and a use of information
theory \citep{ulanowicz86, ulanowicz97, ulanowicz04}.  The second
school has an environment focus and is built on the environ concept
introduced by Dr.\ Bernard C.\ Patten \citep{patten76, patten78,
  fath99_review}.  Patten's approach has been collectively referred to
separately as \emph{Network Environ Analysis}. At the core the two
approaches are very similar; however, they make some different
starting assumptions and follow independent yet braided development
tracks. One example difference that has historically inhibited
collaboration and applications is that the two schools orient their
analytical matrices in different ways.  The Ulanowicz school orients
their matrices as flows from rows-to-columns, which is the most common
orientation in the broader field of network science
\citep[e.g.,][]{brandes05}.  In contrast, the Patten School has
historically oriented their matricies from column-to-row.  Recent
research has started to bring the work of the two schools back
together \citep[e.g.,][]{scharler09comparing}; we hope this software
contributes to this.  \citet{borrett12_netecol} provides an entry
level overview of the field.

% objectives
Disparate software packages have been created to support
ENA. Ulanowicz first developed and distributed the DOS based NETWRK4
code, which is still available.  Recently some of these algorithms were
reimplemented in an Microsoft Excel based WAND package
\citep{allesina04_wand}.  Some of these methods have also been encoded
in the popular Ecopath with Ecosim software that assists with model
construction \citep{christensen04}.  \citet{fath06} published NEA.m, a
MATLAB\copyright function that collected the Patten School's
algorithms together into one set of code.  One objective for this
\R\ package is to begin to bring together these different
algorithms into a single accessible and extensible package.  The
primary ENA algorithms included in this package are summarized in
Table~\ref{tab:alg} and a plot of the network of functions for the
package can be found in Figure~\ref{fig:fig}.

\begin{table*}
\center
\caption{Primary Ecological Network Analysis algorithms in
  \textit{enaR}.} \label{tab:alg}
\tableline
%\begin{scriptsize}
\begin{tabular}{l l l l l }
\textbf{Analysis} & \textbf{Function Name} & \textbf{School} \\ \hline \\ [-1ex]
Structure & \texttt{enaStructure} & foundational, Patten \\
Flow & \texttt{enaFlow} & foundational, Patten \\
Ascendency & \texttt{enaAscendency} & Ulanowicz \\
Storage & \texttt{enaStorage} & Patten \\
Utility & \texttt{enaUtility} & Patten \\
Mixed Trophic Impacts & \texttt{enaMTI} & Ulanowicz \\
Control & \texttt{enaControl} & Patten \\
Environ & \texttt{enaEnviron} & Patten \\
\end{tabular}
%\end{scriptsize}
\tableline
\end{table*}





